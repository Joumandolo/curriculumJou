%% start of file `template_en.tex'.
%% Copyright 2007 Xavier Danaux (xdanaux@gmail.com).
%
% This work may be distributed and/or modified under the
% conditions of the LaTeX Project Public License version 1.3c,
% available at http://www.latex-project.org/lppl/.


\documentclass[11pt,a4paper,spanish]{moderncv}

% moderncv themes
%\moderncvtheme[blue]{casual}                 % optional argument are 'blue' (default), 'orange', 'red', 'green', 'grey' and 'roman' (for roman fonts, instead of sans serif fonts)
\moderncvtheme[blue]{casual}                % idem

% character encoding
\usepackage[utf8]{inputenc}                   % replace by the encoding you are using

% adjust the page margins
\usepackage[scale=0.85]{geometry}

\recomputelengths                             % required when changes are made to page layout lengths

% personal data
\firstname{Manuel Arredondo Bravo}
\familyname{}
\title{Ingeniero Comercial}        % optional, remove the line if not wanted%\address{Talasia 550 Jardín del Mar,}{Viña del Mar}    % optional, remove the line if not wanted
\mobile{+56 09 0163753}                            % optional, remove the line if not wanted
\phone{+56 32 3173441}                    % optional, remove the line if not wanted
\email{m.arredondo@vtr.net}
%\photo[80pt]{1.jpg}

\nopagenumbers{}                             % uncomment to suppress automatic page numbering for CVs longer than one page

%----------------------------------------------------------------------------------
%            content
%----------------------------------------------------------------------------------

\begin{document}
\maketitle

\section{Datos Personales}
% \cventry{year--year}{Degree}{Institution}{City}{\textit{Grade}}{Description}  % arguments 3 to 6 are optional
\cventry{RUN}{7.552.293-2}{}{}{}{}
\cventry{Fecha de Nacimiento}{15 de Julio de 1956}{}{}{}{}
\cventry{Nacionalidad}{Chilena}{}{}{}{}
\cventry{Estado Civil}{Casado}{}{}{}{}
\cventry{Hijos}{Cinco}{}{}{}{}
\cventry{Direcci\'on}{Talasia 550, Jard\'in del Mar, Vi\~na del Mar}{}{}{}{}
\cventry{Fono Hogar}{+56 32 3173441}{}{}{}{}
\cventry{Fono M\'ovil}{+56 09 0163753}{}{}{}{}
\cventry{E-Mail}{m.arredondo@vtr.net}{}{}{}{}

\section{Motivación laboral y docente}
\cvlistitem{La principal motivación y experiencia laboral está     orientada a administrar y liderar procesos de cambio, reestructuración organizacional y procesos de puesta en marcha, como asimismo, la formación y desarrollo de proyectos nuevos.}
\cvlistitem{Desde la experiencia y por la realización del Magister en Gestión y Liderazgo Educacional, existe una real motivación por la docencia, en las áreas de administración, liderazgo organizacional, evaluación y puesta en marcha de proyectos.}

\section{Ultimo lugar de trabajo}
\cventry{2001 - 31 Julio 2009}{\textbf{Colegio de los Sagrados Corazones de Valparaíso-Viña del Mar}}{Avenida Padre Hurtado 1520, Viña del Mar, Chile}{Telefono: +56 32 2387438}{Email: rectoria@colegiosscc.cl}{}
\cventry{Funciones}{Vicerector de Administración, Sub Director del proyecto Nuevo Colegio}{Esta gestión implicó administrar la fusión de dos comunidades educacionales con 172 años de fundación, 1300 alumnos, 850 familias, 180 personas entre docentes y administrativos, desarrollo del proyecto educativo y formación de equipos de trabajo. En este período se administró dos proceso de convenios colectivos de trabajo, manteniéndose un buen clima laboral. El costo del proyecto (construcción del establecimiento, compra de terreno y puesta en marcha: UF 500.000.-}{}{}{}{}

\section{Experiencia Laboral}
\cventry{1997 - 2000}{\textbf{Colegio Valle del Aconcagua, Quillota, V Región}}{\textit{Socio y gerente}}{Esta gestión consistió en la formación del colegio, desarrollo del proyecto educativo, compra del terreno, la construcción del establecimiento, formación de equipos de trabajo. Proceso que duró un año. La puesta en marcha y gerencia del colegio se desarrollo en los  3 años posteriores. Actualmente cumplió más de una década y tiene 400 alumnos.}{}{}
\cventry{1985 - 2000}{\textbf{Huerto Florence Ltda, Hijuelas, V Región}}{\textit{Socio y gerente}}{Este proyecto consistió en el cultivo de 30 hectáreas de paltos. Su administración y proceso de exportación. Esta exportación (en conjunto con otros productores) fue la primera partida que realizó Chile al exterior (Europa).Un segundo proyecto consistió en el desarrollo de un vivero de cultivos de plantas de cítricos certificados. Este proyecto se desarrollo en conjunto con la Escuela de Agronomía de la PUCV y la Asociación de Viveros IVIA Valencia, España. Ambos proyectos implicaron el desarrollo e implementación, evaluación de costos, evaluación de mercados, negociaciones internas y externas, puesta en marcha y acciones comerciales. Asimismo, la formación de equipos profesionales del ámbito agronómico y empresarios.}{}{}
\cventry{1996 - 2000}{\textbf{Fundación la Semilla,Hijuelas, V Región}}{\textit{Gerente gestor}}{Esta administración consistió en la creación de 4 centros de capacitación para mujeres campesinas, en cuatro diferentes comunas de la V Región interior. Además, el desarrollo del proyecto de educación al aire libre, construyendo una Villa para niños de alto riesgo en Hijuelas.}{}{}
\cventry{1984 - 2000}{Otras actividades}{Ejercicio libre de la profesión, Asesor en recursos humanos en el campo agroindustrial (Propal, Hijuelas; Agronueve, Quillota; Fríomaipo, San Felipe)}{}{}{}

\section{Formación Académica, Título Profesional y Grados}
\cventry{1984}{Ingeniero Comercial}{Universidad de Chile}{}{}{}
\cventry{1982}{Bachiller en Administración}{Universidad de Chile}{}{}{}
\cventry{2008 - a la fecha}{Magister en Gestión y Liderazgo Educacional}{Universidad Alberto Hurtado, Santiago}{Desarrollando Tesis}{}{}
\cventry{1995 - 1998}{Catequista}{Instituto de Agentes Pastorales, IFAP, Obispado de Valparaíso}{}{}{}
%\cventry{1999}{Monitor Teen Star}{Pontificia Univercidad Catolica de Chile}{Monitor de sexualidad holística para jóvenes adolescentes}{}{}
\cventry{1974}{Educación Básica y Media}{Colegio de los Sagrados Corazones Viña del Mar}{}{}{}{}

\closesection{}                   % needed to renewcommands
\renewcommand{\listitemsymbol}{$\bullet$} % change the symbol for lists

\section{Otras Características}
\cvlistitem{Opción familiar, 27 años de matrimonio}
\cvlistitem{Hijos profesionales, universitarios y en etapa escolar.}
\cvlistitem{Desarrollo de una experiencia laboral y profesional como sentido de vida}
\cvlistitem{Con capacidad y motivación en la acción de proyectos, en la formación de equipos de trabajos multidisciplinarios basados en un liderazgo situacional, en la capacidad emprendedora, la libertad y  los resultados.}
\cvlistitem{Estudioso en la formación de la Fe y el crecimiento en la  interioridad personal. Formador de padres y parejas que pretenden un referente trascendente para sus hijos.}
\cvlistitem{La lectura, estar y compartir en familia, trotar y las excursiones deportivas compartidas en la naturaleza.}

\vspace*{1.5cm}
\begin{flushright}
    \textbf{Agosto, 2009}\\[1.5cm]
\end{flushright}

\end{document}
